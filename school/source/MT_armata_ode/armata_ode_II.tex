\documentclass[10pt,twocolumn]{article}

\usepackage{geometry}
\geometry{a4paper}
%%\usepackage[cm]{fullpage}

%%\usepackage[top=2in, bottom=1.5in, left=2in, right=1in]{geometry}

%% \usepackage[active,tightpage]{preview}
%% \PreviewEnvironment{eqnarray}
%% \PreviewEnvironment{equation}

\addtolength{\hoffset}{-25pt}
\addtolength{\textwidth}{35pt}


\usepackage{graphicx}
\usepackage{float}
%\usepackage{wrapfig}

\linespread{1.} % Line spacing

\graphicspath{{figs/}}

\usepackage{polski}
\usepackage[utf8]{inputenc}

\DeclareFixedFont{\ttb}{T1}{txtt}{bx}{n}{8} % for bold
\DeclareFixedFont{\ttm}{T1}{txtt}{m}{n}{8}  % for normal
\usepackage{color}
\definecolor{deepblue}{rgb}{0,0,0.5}
\definecolor{deepred}{rgb}{0.6,0,0}
\definecolor{deepgreen}{rgb}{0,0.5,0}
\usepackage{listings}


% Python style for highlighting
\newcommand\pythonstyle{\lstset{
language=Python,
basicstyle=\ttm,
otherkeywords={self},             % Add keywords here
keywordstyle=\ttb\color{deepblue},
emph={MyClass,__init__},          % Custom highlighting
emphstyle=\ttb\color{deepred},    % Custom highlighting style
stringstyle=\color{deepgreen},
frame=tb,                         % Any extra options here
showstringspaces=false            %
}}
% Python environment
\lstnewenvironment{python}[1][]
{
\pythonstyle
\lstset{#1}
}
{}
% Python for external files
\newcommand\pythonexternal[2][]{{
\pythonstyle

\lstinputlisting[#1]{#2}}}
% Python for inline
\newcommand\pythoninline[1]{{\pythonstyle\lstinline!#1!}}

\title{Strzelamy i latamy, czyli o prawie Newtona i równaniach różniczkowych cześć II}
\begin{document}
\maketitle
\lstset{language=Python}

Na lekcjach fizyki w szkole pewnie wszyscy spotkali się ze wzorem na
prędkość: $v=\frac{\Delta x}{\Delta t}$ oraz przyspieszenie
$a=\frac{\Delta v}{\Delta t}$. Pokażemy, że umiejętne skorzystanie z
tych pozornie mało spektakularnych wzorów umożliwi napisanie programu
komputerowego, który będzie mógł rozwiązywać praktycznie
nieograniczoną pulę zagadnień z dynamiki.

Odkrywania praw fizycznych poprzez ``komputerowe eksperymentowanie'' z
równaniami jest celem projektu dydaktycznego iCSE prowadzonego na
Uniwersytecie Śląskim. Jako narzędzie proponujemy system
Sage\ \cite{sagemath}. Stanowi on otwartoźródłową integrację systemu
algebry komputerowej z językiem Python, ponadto umożliwia rozpoczęcie
zabawy od zaraz, korzystając z przeglądarki internetowej i jednej z
możliwych opcji dostępu poprzez usługę chmurową\ \cite{cloud} lub
serwer pojedynczych obliczeń, na którym bazuje interaktywna wersja
tego artykułu\ \cite{web}.



\section{Równanie Newtona}

W poprzednim artykule dowiedzieliśmy się, że można rozwiązywać
zagadnienia z zakresu dynamiki Newtona za pomocą metody przybliżonej
stosując znane ze szkoły wzory na prędkość w ruchu jednostajnym
prostoliniownym oraz przyśpieszenie w ruchu jednostajnie
przyśpieszonym. W przypadku dwuwymiarowego zagadnienia, którym jest
rzut ukośny, stan układu jest jednoznacznie określony przez cztery
liczby: $x,y,v_x,v_y$. Równania wiążące stan układu w pewnej chwili
poźniejszej $t+\Delta t$ ze stanem w chwili $t$ mają postać:

\begin{eqnarray}
x(t+\Delta t)&=& x(t)+ \Delta t \cdot v_x \\ \nonumber
y(t+\Delta t)&=& y(t)+ \Delta t \cdot v_y \\ \nonumber
v_x(t+\Delta t) &=&v_x(t) + \Delta t\cdot  \frac{F_x}{m}\\ \nonumber
v_y(t+\Delta t) &=&v_y(t) + \Delta t\cdot  \frac{F_y}{m},
\label{eq:euler4} 
\end{eqnarray}
gdzie $F_x$ i $F_y$ są siłami działającymi na punkt materialny o masie
$m$. Jeżeli na ciało działa tylko siła grawitacyjna to mamy $F_x=0$ i
$F_y=-mg$.

Równania te wyprowadzilismy w poprzednim artykule, korzystając ze
wzorów na przyśpieszenie i prędkość, napisaliśmy składowe równania
Newtona w kierunkach $x$ i $y$ jako:
 
\begin{eqnarray}
\frac{ \Delta x}{\Delta t} &=& v_x \\ \nonumber
\frac{ \Delta y}{\Delta t} &=& v_y \\ \nonumber
\frac{ \Delta v_x}{\Delta t} &=& \frac{F_x}{m}\\ \nonumber
\frac{ \Delta v_y}{\Delta t} &=& \frac{F_y}{m},
\label{eq:delta} 
\end{eqnarray}

Takie równania znane są w granicy gdy $\Delta t\to 0$ jako równania
różniczkowe. Granica ta oznacza, że wyrażenia typu $ \frac{ \Delta
  x}{\Delta t}$, określane przez matematyków jako ilorazy różnicowe,
przechodzą w pochodne po czasie funkcji położenia czy
prędkości. Intuicyjnie, możemy zrozumieć tą granicę jeśli sobie
przypomnimy, że w układzie (\ref{eq:delta}) zastosowaliśmy przybliżone
a nie dokładne wzory na prędkość. Błąd tego przybliżenia maleje do
zera jeśli $ \Delta t\to 0$. W tej granicy równania te są zapisywane,
zastępując symbole $\Delta$ symbolami $d$:

\begin{eqnarray}
\frac{ dx}{dt} &=& v_x \\ \nonumber
\frac{ dy}{dt} &=& v_y \\ \nonumber
\frac{ dv_x}{dt} &=& \frac{F_x}{m}\\ \nonumber
\frac{ dv_y}{dt} &=& \frac{F_y}{m},
\label{eq:ode} 
\end{eqnarray}

Nie wnikają już głębiej w fascynującą dziedzinę matematyki opisującą w
ścisły sposób tego typu równania, postarajmy się teraz wykorzystać
znajomość algorytmu przybliżonego i rozwiązać ciekawy  układ.


\section{Modelujemy papierowy samolot }

Puszczając papierowy samolot można zauważyć, że czasem zadziera on do
góry a czasem leci lotem prostoliniowym. Od czego to zależy?
Doświadczenia sugerują, że może prędkość początkowa i tzw. trym, czyli
ustawienie lotek może być decydującym czynnikiem. Jaka jest jednak
fizyka tego zjawiska?

Papierowy samolot - cóż on ma wspólnego z rzutem ukośnym? Przecież w
rzucie ukośnym mamy do czynienia z punktem materialnym a tu jest
zdecydowanie rozciągły objekt, do tego jeszcze oddziaływuje w
nietrywialny sposób z powietrzem! Zachęcam czytelników do zapytania
przypadkowo napotkanego naukowca o to jak obliczyć siły działające na
papierowy samolot. Z dużym prawdopodobieńswtem dowiecie się, że jest
to problem przekraczający możliwości nauki ....

Możemy jednak spróbować dokonać pewnych idealizacji. Po pierwsze
trzeba się zastanowić jak podejdziemy do oddziaływania powietrza na
samolot. Możemy rozwiązać niezwykle trudne do analizy równania
aerodynamiki. Zdecydowanie przekroczyło by to ramy tego
artykułu. Jakie więc mamy inne możliwości? Możemy skorzystać z pewnych
praw z znanych z mechaniki lotu. Niestety mechaniki lotu nie uczy się
w szkołach a prawa nie mają charakteru fundamentalnych praw przyrody a
raczej są rozsądnymi przybliżeniami rzeczywistości. W aerodynamice
wyróżnia się dwie podstawowe siły działające na objekt poruszający się
w powietrzu: siłę oporu oraz siłe nośną. Ta pierwsza jest nam dobrze
znana z rozważań lotu pocisku przez powietrze. Wiemy, że działa ona w
kierunku przeciwnym do wektora prędkości. Druga siła jest siłą
działającą w kierunku prostopadłym do wektora prędkości i jest zwana
siłą nośną. Mechanika lotu podaje empiryczne wzory pozwalające
obliczyć te siły dla dobrze opływanych objektów, do których zaliczaja
się skrzydło samolotu. Oba wzory, zarówno na siłę nośną i opór,
zakładają kwadratową zależność od prędkości.

Łatwo można sobie wyobrazić, że na skrzydło ustawione równolegle do
napływającego powietrza będzie działała niewielka siła oporu a siła
nośna będzie bliska zeru. Jeśli skrzydło przekręcimy o np. 5 stopni,
to siła nośna znacznie wzrośnie. Siła oporu też wzrośnie, ale mniej
radykalnie niż siła nośna. Zjawisko to można zaobserwować wychylając
dłoń z jadącego samochodu i obracając ją o niewielki kąt.

Prawa mechaniki lotu określają, współczynniki siły nośnej i oporu oraz
ich zależność od kątą natarcia - czyli ustawienia skrzydła względem
napływających strug powietrza. Oznacza to, że jeśli mamy dany rozmiar
i kształt skrzydła to dla każdego kąta natarcia oraz prędkości możemy
wyliczyć obie siły. Za chwilę skorzystamy z tego faktu.

Wyobraźmy teraz sobie szybowiec, a papierowy samolocik jest właśnie
szybowcem, lecący z pewną prędkością. Po pierwsze zakładamy,
że nie wykonujemy zakrętów. Dopuszczamy tylko nurkowanie i wznoszenie
się. W pewnym przybliżenu można traktować szybowiec jako skrzydło,
które ma pewną siłę nośną i siłę oporu oraz ogon. Zadaniem ogona jest
utrzymanie skrzydła pod pewnym kątem do nacierającego powietrza. Jak
on to robi? Zawsze stara się tak ustawić by stateczniki poziome
(pionowe nas nie interesują, bo nie skręcamy) były równoległe do
wektora prędkości. Ustawiając się przekręca nieco cały samolot. Jako,
że ogon znajduję się daleko z tyłu to dysponuje dość dużą dzwignią. W
naszym modelu zakładamy idealny ogon - jest bardzo maly i bardzo
daleko od skrzydeł, tak by siły na niego działające były zaniedbywalne
przy siłach działających na skrzydło, ale moment siły był na tyle duży
by samolot ustawiał się natychmiastowo w pozycji - ``ogon idealnie z
tylu''. Sterowanie kierunkiem lotu odbywa się poprzez zmianę kąta
zaklinowania satecznika poziomego. Jeśli pilot pociągnie drążek do
siebie to ustawia sie on w dół co powoduje taki obrót calego samolotu
by ``ogon był równoległy do prędkości''. Zauważmy, że skrzydło
samolotu jest ustawione wzgledem lini łączącej jego środek z ogonem
pod pewnym katem. Jest to tzw. kąt zaklinowania skrzydła. Z reguły ma
on małą dodatnią wartość. Jeżeli teraz będziemy rozważać lot szybowca
dla pewnego stałego ustawienia drążka sterowego. Papierowy samolot
jest rówież w niezłym przybliżeniu takim właśnie przypadkiem.

W takiej sytuacji będziemy mieli zagwarantowane, że kąt natarcia
skrzydła będzie zawsze ten sam. Wtedy będziemy mieli w ciągu całego
lotu, niezależnie od tego czy samolot nurukje czy się wznosi, proste
formuły na siły aerodynamiczne. Siła nośna będzie równa $C_z v^2$ a
siła oporu $C_x v^2$. Stałe $C_x,C_z$, są zwane współczynnikami sił,
odpowiednio, oporu i nośnej. W takim przypadku możemy sie pokusić o
zmodyfikowanie, naszego programu do obliczania trajektorii rzutu
ukośnego tak by opisywał on zachowanie się naszego szybowca.

Ponieważ siły aerodynamiczne działają w kierunkach prostopadłych i
równoległych do kierunku ruchu to musimy przeprowadzić sztuczkę
zamiany układu współrzędnych. Postąpimy tak: znamy wektor prędkości
samolotu. Dzieląc go przez jego długośc możemy otrzymać wersor (wektor
jednostkowy) w kierunku ruchu $\vec e_v$. 

Siły aerodynamiczne: oporu $F_D$ i nośna $F_L$ dane są:

\begin{eqnarray}
F_D = - C_D v^2  \vec e_v \\
F_L = C_L v^2 {\vec e_v}_{prost}

\end{eqnarray}

Zauważmy, że łatwo możemy otrzymać wersor prędkości - wys





\section{Czy możemy obliczyć donośność armaty  Flak 88m?}

\pythonexternal{code04.py}



\begin{thebibliography}{1}
\bibitem{sagemath} http://sagemath.org
\bibitem{cloud} http://cloud.sagemath.com
\bibitem{web} http://visual.icse.us.edu.pl/Warsztaty

\end{thebibliography}

\end{document}


